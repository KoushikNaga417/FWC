\documentclass{article}
\usepackage{amsmath}
\usepackage{enumerate}
\usepackage{enumitem}
\usepackage{tfrupee}
\usepackage{float}
\usepackage{textcomp,gensymb}
\usepackage{bbm}
\begin{document}
\providecommand{\mydet}[1]{\ensuremath{\begin{vmatrix}#1\end{vmatrix}}}
\providecommand{\myvec}[1]{\ensuremath{\begin{bmatrix}#1\end{bmatrix}}}
\providecommand{\brak}[1]{\ensuremath{\left(#1\right)}}
\providecommand{\cbrak}[1]{\ensuremath{\left\{#1\right\}}}
\begin{enumerate}
\section {Matrices}
             \item \noindent Find matrix A such that $ 2A - 3B + 5C = O $,where B = $\myvec{-2 & 2 & 0 \\3 & 1 & 4 }$ and C= $\myvec{2&0&-2\\7&1&6 }$
             \item If $A=\myvec {1&2&-2 \\-1 &3&0\\3&1&1}$ , find $A{-1}$ . Hence solve the following system of equations $x+y+z= 6,x+2z=7,3x+y+z=12$. 
	     \item Find the inverse of the following matrix using elementary operations \begin{align*}\myvec{1&2&-2\\-1&3&0\\3&1&1}\end{align*}
		     \item Find the value of x-y,if \begin{align*}\myvec{1&3\\0&x}+\myvec{y&0\\1&2}=\myvec{5&6\\1&8}.\end{align*} 
              \item If $A$ and $B$ are square matrices of the same order 3, such that $\mydet{A} = 2$ and $AB = 2I$, write the value of$\mydet{B}$.
              \item Using properties of determinants,prove that $\mydet{a^2+2a & 2a+1 & 1\\2a +1& a+2 & 1\\3 & 3 & 1}= (a-1)^3$
              \item Using the properties of determinants,prove the following:\[\mydet{a+b+c&-c&-b\\-c&a+b+c&-a\\-b&-a&a+b+c}=2(a+b)(b+a)(c+a)\]             
\section{Relations and functions}
           \item If $f(x) = x + 1$, find $\frac{d}{dx}\brak{f \circ f}\brak{x}$
	   \item Examine whether the operation $*$ defined on R by $a * b = ab + 1$ $\brak{i}$is a binary operation.$\brak{ii}$ If a binary operation, is it associative or not?
            \item prove that the function$f:N \rightarrow N$,defined by $f(x)=x^2+x+1$ is one-one but not onto.Find the inverse of $f:N\rightarrow S$,where S is range of f.
            \section{Algebra}
\item Solve:$\tan^{-1}4x+\tan^{-1}6x=\frac{\pi}{4}$
\item If $log\brak{x^2+y^2}=2\tan^{-1}\frac{y}{x}$,show that $\frac{dy}{dx}=\frac{x+y}{x-y}$
\item If $x^y-y^x=a^b$,find $\frac{dy}{dx}$
\item If $y=\brak{\sin^{-1}x}^2$,prove that $(1-x^2)\frac{d^2y}{dx^2}-x\frac{dy}{dx}-2=0$
\item Find:$\int\dfrac{\tan^2x\sec^2x}{1-\tan^6x},dx$
\item Solve for x:$\tan^{-1}(2x)+\tan^{-1}(3x)=\frac{\pi}{4}$
\item If $x=\cos t+log \tan\brak{\frac{t}{2}},y=\sin t$,then find the values of $\frac{d^2y}{dt^2}$ and $\frac{d^2y}{dx^2}$ at $t=\frac{\pi}{4}$
\section{Differentiation}
	\item Find the order and the degree of the differential equation $x^2\frac{d^2y}{dx^2}=\cbrak{1\brak{\frac{dy}{dx}}^2}^4$ 
        \item Form the differential equation representing the family of curves $y = e^{2x}\brak{a + bx}$, where 'a' and 'b' are arbitrary constants.
\section{Vectors} 
\item If the sum of two unit vectors is a unit vector,prove that the magnitude of their difference is $\sqrt{3}$.
\item If $\overrightarrow{a}=2\hat{i}+3\hat{j}+\hat{k}$, $\overrightarrow{b}=\hat{i}-2\hat{j}+\hat{k}$ and $\overrightarrow{c}=-3\hat{i}+\hat{j}+2\hat{k}$,Find $\myvec{\overrightarrow{a}&\overrightarrow{b}&\overrightarrow{c} }$
\item If $\hat{i}+\hat{j}+{k} ,  2\hat{i}+5\hat{j} ,  3\hat{i}+2\hat{j}-3{k} ,  \hat{i}-6\hat{j}-{k}$ respectively are the position vectors of points A, B, C and D, then find the angle between the straight lines AB and CD. Find whether $\overrightarrow{AB}$ and $\overrightarrow{CD}$ are collinear or not. 
\item Find the vector equation of the line which passes through the point $\brak{3,4,5}$ and is parallel to the vector $2\hat{i}+2\hat{j}-3\hat{k}$
\item Find the vector and Cartesian equations of the plane passing through the points $\brak{2, 2 –1}, \brak{3, 4, 2}$ and $\brak{7, 0, 6}$. Also find the vector equation of a plane passing through $\brak{4, 3, 1}$ and parallel to the plane obtained above.
\item Find the vector equation of the plane that contains the lines $r = \brak{\hat{i}+\hat{j}}+\lambda \brak{\hat{i}+ 2\hat{j} –\hat{k}} $and the point $\brak{–1, 3, – 4}$. Also,find the length of the perpendicular drawn from the point (2, 1, 4) to the plane, thus obtained .
\item Find the value of $\lambda$, so that the lines $\frac{1-x}{3}=\frac{7y-14}{\lambda}=\frac{z-3}{2}$ and $\frac{7-7x}{3\lambda}=\frac{y-5}{1}=\frac{6-z}{5}$ are at right angles. Also, find whether the lines are intersecting or not.       
\item If a line makes angles $ 90\degree,135\degree,45\degree$ with the x,y and z axes respectively,find its direction cosines.
\section{Integration}
\item  Find: $\int \dfrac{\sec^2x}{\sqrt{\tan^2x+4}},dx$
\item Find: $\int\sin^{-1}2x,dx $
\item \noindent Find: $\int\sqrt{1-\sin2x},dx$ , $\frac{\pi}{4}< x < \frac{\pi}{2}$
\item Find: $ \int\sin^{-1}2x,dx$
\item Find: $\int\dfrac{3x+5}{x^2+3x-18},dx$
\item prove that $\int_{0}^{a}f(x),dx=\int_{0}^{a}f(a-x),dx$ , hence evaluate $\int_{0}^{\pi}\frac{x\sin x}{1+\cos^2x}dx$ .
\item Solve the differential equation :$\brak{1+x^2}\frac{dy}{dx}+2xy-4x^2=0$ ,subject to the initial condition $y\brak{0}=0$ 
\item Solve the differential equation:$xdy-ydx=\sqrt{x^2+y^2}dx$ ,given that $y=0$ when $x=1$.
\section{Intersection of Conics}
\item Using integration, find the area of triangle ABC, whose vertices are A$\brak{2, 5}$, B$\brak{4, 7}$ and C$\brak{6, 2}$.
\item Find the area of the region lying above x-axis and included between the circle $x^2+y^2 = 8x$ and inside of the parabola $ y^2 = 4x$
\item Find the equation of tangent to the curve $y=\sqrt{3x – 2} $ which is parallel to the line $ 4x – 2y + 5 = 0$. Also, write the equation of normal to the curve at the point of contact.
\section{Probability}
\item A die marked $1, 2, 3 $ in red and $4,5,6 $ in green is tossed. Let A be the event “number is even” and B be the event “number is marked red. Find whether the events A and B are independent or not.
\item A die is thrown $6$ times. If “getting an odd number” is a “success”, what is the probability of (i) $5$ successes ? (ii) atmost $5$ successes ?
\item A manufacturer has three machine operators A, B and C. The first operator A produces $1\%$ of defective items, whereas the other two operators B and C produces $5\%$ and $7\%$ defective items respectively. A is on the job for $50\%$ of the time, B on the job $30\%$ of the time and C on the job for $20\%$ of the time. All the items are put into one stockpile and then one item is chosen at random from this and is found to be defective. What is the probability that it was produced by A ?
\item The random variable $X$ has a probability distribution $P\brak{X}$ of the following form, where $k$ is some number:
\[P(X = x) = \begin{cases}
    k, & \text{if } x = 0 \\
    2k, & \text{if } x = 1 \\
    3k, & \text{if } x = 2 \\
    0, & \text{otherwise}
\end{cases}\] 
Determine the value of $k$.
\section{Optimization}
\item A manufacturer has employed 5 skilled men and 10 semi-skilled men and makes two models A and B of an article. The making of one item of model A requires 2 hours work by a skilled man and 2 hours work by a semi-skilled man. One item of model B requires 1 hour by a skilled man and 3 hours by a semi-skilled man. No man is expected to work more than 8 hours per day. The manufacturer’s profit on an item of model A is \rupee~$15$ and on an item of model B is \rupee~$10$. How many of items of each model should be made per day in order to maximize daily profit ? Formulate the above LPP and solve it graphically and find the maximum profit.
\item A tank with rectangular base and rectangular sides, open at the top is to be constructed so that its depth is 2 m and volume is $8 m^3$ . If building of tank costs \rupee~${70}$ per square metre for the base and \rupee~${45}$ per square metre for the sides, what is the cost of least expensive tank ?
\end{enumerate}
\end{document}
