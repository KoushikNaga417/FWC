\documentclass{article}
\usepackage{graphicx}
\usepackage{enumitem}
\usepackage{circuitikz}
\providecommand{\brak}[1]{\ensuremath{\left\{#1\right\}}}
\begin{document}
\begin{enumerate}
\item The state diagram of a sequence detector is shown below. State $S_{0}$ is the initial state of the sequence detector. If the output is 1, then
\begin{circuitikz}
    % Input nodes
    \node (a) at (0,0) {$a$};
    \node (b) at (0,-1) {$b$};
    \node (c) at (0,-3) {$c$};

    % NOT gate
    \node [not port] at (2,-1) (not) {};
    \draw (b) -- (not.in);

    % AND gate
    \node [and port] at (5,-2.75) (and) {};
    \draw (not.out) -| (and.in 1);
    \draw (c) -- (and.in 2);

    % OR gate
    \node [or port] at (8,-0.45) (or) {};
    \draw (a) -- (or.in 1);
    \draw (and.out) -| (or.in 2);

    % Output
    \node (z) at (9,-0.5) {$z$};
    \draw (or.out) -- (z);
\end{circuitikz}
\begin{enumerate}[label=\Alph*.]
 \item the sequence 01010 is detected
 \item the sequence 01011 is detected
 \item the sequence 01110 is detected
 \item the sequence 01001 is detected
\end{enumerate}
\end{enumerate}
\hfill (GATE CS 2020)
\end{document}
